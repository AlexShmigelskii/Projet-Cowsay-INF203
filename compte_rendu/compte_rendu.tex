\documentclass[a4paper,french,11pt]{article}
\usepackage[french]{babel}
\usepackage[utf8]{inputenc}
\usepackage{titlesec}
\usepackage{fancyhdr}
\usepackage{listings} 
\usepackage{xcolor}
\usepackage{geometry}
\usepackage[T1]{fontenc}
\geometry{hmargin=2.5cm,tmargin=1.5cm, bmargin=3cm}
\setlength\hoffset{1em}
\setlength\headheight{32pt}
\setlength\parindent{1em}
\usepackage{hyperref}
\hypersetup{
	colorlinks=true,
	linkcolor=black, 
	urlcolor=[rgb]{0.05,0,0.25}
	}

\lstset{
	numbers=left,
	numberstyle=\tiny\color{gray},
	numbersep=5pt,
	language=bash,
	basicstyle=\ttfamily\footnotesize,
	showstringspaces=false,
	keywordstyle=\bfseries\color{olive},
	% commentstyle=\itshape\color{gray},
	stringstyle=\color{blue},
	breaklines=true,
	breakatwhitespace=true,
	tabsize=4
	}

\titleformat{\section}{\bfseries\Large}{\thesection.}{0.5em}{}
\titleformat{\subsection}{\bfseries\large}{\hspace{.5em}\thesubsection.}{0.5em}{}


%%%%%%%%%%%%%%
%% pied de page et entête
%
\pagestyle{fancy}
\renewcommand{\headrulewidth}{0pt}
\fancyhead{\empty}
\renewcommand{\footrulewidth}{.5pt}
\fancyfoot[L]{\sffamily INF203, Projet Cowsay\empty}
\fancyfoot[C]{\sffamily Shmigelskii A., MELLA G., BASS D.}
\fancyfoot[R]{\sffamily Compte Rendu, Page \thepage}
%%%%%%%%%%%%%%


\begin{document}
\thispagestyle{empty}
\begin{center}

\vspace{10ex}
\huge\sffamily\textbf{Compte Rendu du Projet Cowsay}

\vspace{3ex}
\large


\vspace{3ex}
Aleksandr Shmigelskii, Gabriel Mella, Daniel Bass \\ IMA-05
\end{center}
\vspace{2ex}
{\sffamily \large}

\vspace{.5ex}
\noindent{Enseignant de TD/TP:} Jean-Loup Haberbusch

\vspace{.5ex}
\tableofcontents  
\vspace{2ex}

\newpage
\section{Introduction}

\newpage
\section{Préliminaires}

\begin{table}[ht]
	\centering
	\begin{tabular}{|l|p{5cm}|p{8cm}|}
	\hline
	\textbf{Option} & \textbf{Signification / Effet} & \textbf{Exemple d’usage} \\
	\hline
	\texttt{-b} & Borg mode : la vache aura un aspect “cyborg”. & \verb|cowsay -b "Je suis un Borg"| \\ 
	\hline
	\texttt{-d} & Dead mode : la vache a des yeux « XX ». & \verb|cowsay -d "Aïe. Je ne me sens pas bien"| \\
	\hline
	\texttt{-g} & Greedy mode : la vache a des yeux « \$\$ ». & \verb|cowsay -g "J'adore l'argent"| \\
	\hline
	\texttt{-p} & Paranoïd mode : la vache a des yeux « @@ ». & \verb|cowsay -p "Je suis surveillé..."| \\
	\hline
	\texttt{-s} & Stoned mode : la vache a des yeux « ** ». & \verb|cowsay -s "Coucou..."| \\
	\hline
	\texttt{-t} & Tired mode : la vache a des yeux « -- ». & \verb|cowsay -t "Je suis épuisée..."| \\
	\hline
	\texttt{-w} & Wired mode : la vache a des yeux « OO ». & \verb|cowsay -w "Je ne tiens plus en place"| \\
	\hline
	\texttt{-y} & Youthful mode : la vache a des yeux « .. ». & \verb|cowsay -y "Je suis toute jeune"| \\
	\hline
	\texttt{-e} \emph{eyes} & Personnalise les yeux (2 caractères). & \verb|cowsay -e ^o "Regarde mes yeux"| \\
	\hline
	\texttt{-T} \emph{tongue} & Personnalise la langue (1 ou 2 caractères). & \verb|cowsay -T "U" "Ma langue est sortie"| \\
	\hline
	\texttt{-f} \emph{cowfile} & Utilise un autre dessin ASCII (fichier .cow). & \verb|cowsay -f small "Vraiment petite!"| \\
	\hline
	\texttt{-r} & Choisit une vache au hasard (fichier .cow). & \verb|cowsay -r "Je suis une vache aléatoire."| \\
	\hline
	\texttt{-l} & Liste les vaches définies dans le chemin \textbf{COWPATH} & \verb|cowsay -l| \\
	\hline
	\end{tabular}
	\caption{Principales options de \texttt{cowsay}}
	\label{tab:cowsay-options}
	\end{table}


\newpage
\section{Bash}

\subsection{cow\_kindergarten}

\paragraph{Fonctionnalité}  
Ce script fait « dire » à la vache les nombres de 1 à 10 de manière animée :
\begin{itemize}
  \item \texttt{clear} efface l’écran avant chaque itération pour simuler une animation.
  \item \texttt{cowsay \$i} affiche le chiffre courant (\texttt{i} parcourant \{1…10\}).
  \item \texttt{sleep 1} introduit une pause d’une seconde entre chaque affichage.
  \item À la fin, \texttt{cowsay -T U "J’ai terminé!"} fait tirer la langue à la vache.
\end{itemize}

\paragraph{Exemple d’exécution}  
\begin{verbatim}
$ ./cow_kindergarten.sh
 ___
<  >
 ---
        \   ^__^
         \  (oo)\_______
            (__)\       )\/\
                ||----w |
                ||     ||
...
 ___
< 9 >
 ---
        \   ^__^
         \  (oo)\_______
            (__)\       )\/\
                ||----w |
                ||     ||
 ____________
<J'ai terminé!>
 ------------
     \  ^__^
      \ (oo)\_______  
        (__)\\       )\/\
         U  ||----w |
            ||     ||
\end{verbatim}

\paragraph{Commentaires}  
\begin{itemize}
	\item Utilisation d’une boucle \texttt{for i in \{1..10\}} appelant \texttt{cowsay} à chaque itération.
	\item \textbf{Effet d’animation} : \texttt{clear} + \texttt{sleep 1} suffisent, pas besoin d’outils externes.  
\end{itemize}

\medskip
\noindent\emph{Le code source est fourni dans l’archive (\texttt{scripts/cow\_kindergarten.sh}).}

\newpage

\subsection{cow\_primaryschool}

\paragraph{Différences principales par rapport à \texttt{cow\_kindergarten}}  
\begin{itemize}
  \item Le nombre d’itérations est désormais fixé par l’argument \texttt{\$1}.  
  \item Vérification qu’un seul argument est fourni et qu’il est strictement positif :
    \begin{itemize}
      \item si \texttt{\$\# -ne 1}, on affiche un message d’usage et on quitte ;
      \item si \texttt{\$1 -le 0}, on affiche via \texttt{cowsay} « Veuillez fournir un nombre entier positif supérieur à 0 » puis on quitte.
    \end{itemize}
  \item Boucle \texttt{while [ \$CMPT -le \$N ]} remplace la boucle fixe \texttt{\{1..10\}}.
\end{itemize}

\paragraph{Exemples d’exécution}  
\begin{verbatim}
$ ./cow_primaryschool.sh 5
 ___
< 5 >
 ---
    \   ^__^
     \  (oo)\_______
        (__)\       )\/\
            ||----w |
            ||     ||

$ ./cow_primaryschool.sh -5
 ___________________________________
/ Veuillez fournir un nombre entier \
\ positif supérieur à 0.            /
 -----------------------------------
    \   ^__^
     \  (oo)\_______
        (__)\       )\/\
            ||----w |
            ||     ||

$ ./cow_primaryschool.sh 5 7
Usage: ./cow_primaryschool.sh <nombre n>
\end{verbatim}

\paragraph{Commentaires}  
\begin{itemize}
  \item Le test \texttt{[ \$1 -le 0 ]} intercepte les valeurs non-positives et utilise \texttt{cowsay} pour un message d’erreur plus lisible.  
  \item En cas d’erreur, le script quitte immédiatement sans exécuter la boucle principale.  
  \item Le test \texttt{[ \$1 -le 0 ]} couvre les entiers non positifs, mais si \texttt{\$1} n’est pas numérique, Bash renvoie une erreur de syntaxe (operand expected), mais nous avons supposé que seuls des chiffres seraient transmis.
\end{itemize}

\medskip
\noindent\emph{Le code complet est disponible dans l’archive (\texttt{scripts/cow\_primaryschool.sh}).}

\newpage

\subsection{cow\_highschool}

\paragraph{Différences principales par rapport à \texttt{cow\_primaryschool}}  
\begin{itemize}
  \item Au lieu d’énoncer simplement $i$, la vache énonce son carré $i^2$ grâce à :
    \begin{verbatim}
cowsay $((CMPT * CMPT))
    \end{verbatim}
  \item La structure générale (vérification d’argument, boucle, \texttt{clear}, \texttt{sleep}) reste identique.
\end{itemize}

\paragraph{Exemples d’exécution}
\begin{verbatim}
$ ./cow_highschool.sh 10
 ____
< 36 >
----
    \   ^__^
     \  (oo)\_______
        (__)\       )\/\
            ||----w |
            ||     ||

$ ./cow_highschool.sh -5
 ___________________________________
/ Veuillez fournir un nombre entier \
\ positif supérieur à 0.            /
 -----------------------------------
    \   ^__^
     \  (oo)\_______
        (__)\       )\/\
            ||----w |
            ||     ||

$ ./cow_highschool.sh
Usage: ./cow_highschool.sh <nombre n>
\end{verbatim}

\paragraph{Commentaires}  
\begin{itemize}
  \item Le calcul du carré utilise l’arithmétique intégrée de Bash (\texttt{\$((…))}).  
  \item La validation écrite \texttt{[ \$1 -le 0 ]} couvre les valeurs non‑positives, mais un argument non numérique génère une erreur Shell (« operand expected ») non gérée.  
\end{itemize}

\medskip
\noindent\emph{Le script complet se trouve dans l’archive (\texttt{scripts/cow\_highschool.sh}).}

\newpage

\newpage
\subsection{cow\_college}

\paragraph{Fonctionnalité}  
Ce script énonce les termes de la suite de Fibonacci strictement inférieurs à $n$ :
\begin{itemize}
  \item Vérification de l’argument : un entier > 1 est requis (\texttt{[ \$1 -le 1 ]}).  
  \item Initialisation de deux variables \texttt{FIB1=1}, \texttt{FIB2=1}.  
  \item Boucle \texttt{while [ \$FIB1 -lt \$N ]} :
    \begin{itemize}
      \item Affichage de \texttt{cowsay \$FIB1}.  
      \item Calcul du terme suivant via \texttt{NEW=\$((FIB1+FIB2))}, décalage \texttt{FIB1=\$FIB2}, \texttt{FIB2=\$NEW}.  
      \item \texttt{sleep 1} + \texttt{clear} pour l’animation.
    \end{itemize}
  \item Fin marquée par \texttt{cowsay -T U "J'ai terminé!"}.
\end{itemize}

\paragraph{Exemples d’exécution}
\begin{verbatim}
$ ./cow_college.sh 10
 ___
< 1 >
 ---
    \   ^__^
     \  (oo)\_______
        (__)\       )\/\
            ||----w |
            ||     ||
 ...
 ___
< 8 >
 ---
    \   ^__^
     \  (oo)\_______
        (__)\       )\/\
         U  ||----w |
            ||     ||

$ ./cow_college.sh 1
 ___________________________________
/ Veuillez fournir un nombre entier \
\ positif supérieur à 1.            /
 -----------------------------------
    \   ^__^
     \  (oo)\_______
        (__)\       )\/\
            ||----w |
            ||     ||

$ ./cow_college.sh
Usage: ./cow_college.sh <nombre n>
\end{verbatim}

\paragraph{Commentaires}  
\begin{itemize}
  \item On ne stocke que deux variables \texttt{FIB1}, \texttt{FIB2}.
  \item La condition 
    \texttt{while [ \$FIB1 -lt \$N ]} garantit de n’afficher que les termes strictement inférieurs à $N$, 
    et stoppe avant le premier terme $\ge N$.
  \item Validation minimale : un argument non numérique déclenchera une erreur de shell non gérée.
\end{itemize}

\medskip
\noindent\emph{Le code complet est disponible dans l’archive (\texttt{scripts/cow\_college.sh}).}

\newpage

\subsection{cow\_university}

\paragraph{Fonctionnalité}  
Ce script énonce tous les nombres premiers strictement inférieurs à $n$ :
\begin{itemize}
  \item Vérification de la présence d’un argument unique et de sa positivité (\texttt{[ \$1 -le 1 ]}).  
  \item Boucle \texttt{CMPT=2} à \texttt{CMPT<\$N} :
    \begin{itemize}
      \item Initialisation de \texttt{isPrime=1} (on suppose premier).  
      \item Boucle interne \texttt{while [ \$i -lt \$CMPT ]} testant \texttt{\$CMPT \% \$i}.  
      \item Si un diviseur est trouvé, \texttt{isPrime=0} et \texttt{break}.  
      \item Si \texttt{isPrime==1}, appel de \texttt{cowsay \$CMPT}, \texttt{sleep 1}, \texttt{clear}.
    \end{itemize}
  \item Fin de l’exercice marquée par \texttt{cowsay -T U "J’ai terminé!"}.
\end{itemize}

\paragraph{Exemples d’exécution}
\begin{verbatim}
$ ./cow_university.sh 20
 ___
< 2 >
 ---
        \   ^__^
         \  (oo)\_______
            (__)\       )\/\
                ||----w |
                ||     || 

3 … 19  
    \   ^__^
     \  (oo)\_______
        (__)\       )\/\
            ||----w |
            ||     ||

$ ./cow_university.sh -5
 ___________________________________
/ Veuillez fournir un nombre entier \
\ positif supérieur à 1.            /
 -----------------------------------
    \   ^__^
     \  (oo)\_______
        (__)\       )\/\
            ||----w |
            ||     ||
\end{verbatim}

\paragraph{Commentaires}  
\begin{itemize}
  \item Algorithme naïf de test de primalité en $O(n^2)$, testant tous les diviseurs jusqu’à \texttt{CMPT-1}.  
  \item Interruption précoce dès qu’un diviseur est trouvé (\texttt{break}) pour limiter les calculs.  
  \item Validation minimale : un argument non numérique déclenche une erreur de shell non gérée.  
\end{itemize}

\medskip
\noindent\emph{Le script complet est disponible dans l’archive (\texttt{scripts/cow\_university.sh}).}

\newpage

\subsection{smart\_cow}

\paragraph{Fonctionnalité}  
Le script évalue une expression arithmétique simple (addition, soustraction, multiplication, division) passée en argument et affiche le résultat dans les yeux de la vache :
\begin{itemize}
  \item Vérification qu’un seul argument (la chaîne d’expression) est fourni.  
  \item \textbf{Subshell silencieux + redirection} (\texttt{( res=\$((expr)) ) 2>/dev/null}) pour tester la validité de l’expression sans polluer l’écran.  
  \item Inspection du code de retour (\texttt{\$?}) : si $\ne$  0, message d’erreur \texttt{cowsay "Expression invalide : \$expr"} et sortie.  
  \item Re-calcul du résultat hors subshell (\texttt{res=\$((expr))}) pour récupérer la valeur.  
  \item Détermination de la forme des yeux selon la longueur du résultat :  
    \begin{itemize}
      \item 1 chiffre → \texttt{eyes="\${res}-"}  
      \item 2 chiffres → \texttt{eyes="\$res"}  
      \item >2 chiffres → \texttt{eyes="??"} + message d’excuse.  
    \end{itemize}
  \item Affichage final \texttt{cowsay -e "\$eyes" "\$msg"}.
\end{itemize}

\paragraph{Exemples d’exécution}
\begin{verbatim}
$ ./smart_cow.sh "3+11"
 ___________________________
< Le résultat de 3+11 est 14 >
 ---------------------------
    \   ^__^
     \  (14)\_______
        (__)\       )\/\
            ||----w |
            ||     ||

$ ./smart_cow.sh "3+11+"
 ________________________________
< Expression invalide : 3+11+ >
 --------------------------------
    \   ^__^
     \  (oo)\_______
        (__)\       )\/\
            ||----w |
            ||     ||

$ ./smart_cow.sh 3 + 11
Usage: ./smart_cow.sh "<expression>"
\end{verbatim}

\paragraph{Commentaires et difficultés}  
\begin{itemize}
  \item Capturer l’erreur d’arithmétique Bash nécessite un subshell et \texttt{2>/dev/null}, car \texttt{\$((…))} renvoie un code $\ne$ 0 mais affiche aussi un message sur stderr.  
  \item Refaire le calcul hors subshell est la solution la plus simple pour récupérer \texttt{\$res}.  
  \item Pas de test explicite sur les caractères de l’expression : si on entre une chaîne non arithmétique, elle est évaluée à 0 sans message d’erreur.  
  \item Gestion des cas « yeux trop petits » et format dynamique des yeux selon la longueur du résultat.  
\end{itemize}

\medskip
\noindent\emph{Le script complet est disponible dans l’archive (\texttt{scripts/smart\_cow.sh}).}

\newpage

\subsection{crazy\_cow}

\paragraph{Fonctionnalité}  
Ce script applique une suite d’opérations arithmétiques successives, de gauche à droite, à partir d’un nombre initial. Il affiche à chaque étape une vache avec le résultat intermédiaire dans les yeux et change son comportement si un seuil est dépassé.

\begin{itemize}
  \item Le script attend un argument initial suivi d’un ou plusieurs couples \texttt{<opérateur> <valeur>}.
  \item Il utilise \texttt{shift} pour traiter dynamiquement les arguments deux par deux.  
  \item À chaque itération :
  \begin{itemize}
    \item vérification de l'opérateur (\texttt{+}, \texttt{-}, \texttt{*}, \texttt{/}, \texttt{\%}) ;
    \item vérification que la valeur est bien un entier ;
    \item tentative de calcul (\texttt{result \$op val}) avec gestion d’erreur comme dans \texttt{smart\_cow}.
  \end{itemize}
  \item En cas de dépassement du seuil (\texttt{THRESHOLD=100}), la vache devient folle :
  \begin{itemize}
    \item affichage d’un message de délire (\texttt{eyes = ??}, \texttt{@@}, etc.), 
    \item puis mort finale avec l’option \texttt{-d}.
  \end{itemize}
  \item Si le résultat est :
  \begin{itemize}
    \item négatif → yeux \texttt{XX},
    \item nul → yeux \texttt{??},
    \item normal → yeux \texttt{oo}.
  \end{itemize}
\end{itemize}

\paragraph{Exemple d’exécution}  
\begin{verbatim}
$ ./crazy_cow.sh 5 + 2 - 100 + 3 \* 10
...
< Résultat après -93 + 3 : -90 >
        \   ^__^
         \  (XX)\_______

$ ./crazy_cow.sh 50 + 40 + 30
...
< Aaaaarg, c'en est trop ! Je meurs... >
        \   ^__^
         \  (xx)\_______


$ ./crazy_cow.sh 5 + 2 - 100 = + 3 "*" 10 
...
< Opérateur invalide : = >
        \   ^__^
         \  (OO)\_______

$ ./crazy_cow.sh 5 + 2 - 100 + + 3 "*" 10 
< Oups, impossible de calculer : -93 + + >
        \   ^__^
         \  (oo)\_______
\end{verbatim}

\paragraph{Commentaires et difficultés}
\begin{itemize}
  \item Cette version visait à proposer un script plus original, avec un traitement dynamique des arguments via \texttt{shift} dans une boucle.
  \item Nécessité d’échapper l’astérisque \texttt{*} en ligne de commande (\texttt{\textbackslash*} ou \texttt{"*"}), car sinon le shell tente de l’expanser (globbing) en cherchant les fichiers du répertoire courant.
  \item Vérification élémentaire des erreurs d’entrée via un sous-shell et redirection \texttt{2>/dev/null} pour ne pas afficher les messages Bash.
\end{itemize}

\medskip
\noindent\emph{Le script complet est fourni dans l’archive (\texttt{scripts/crazy\_cow.sh}).}

\end{document}
