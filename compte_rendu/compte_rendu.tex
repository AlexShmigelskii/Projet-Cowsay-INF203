\documentclass[a4paper,french,11pt]{article}
\usepackage[french]{babel}
\usepackage[utf8]{inputenc}
\usepackage{titlesec}
\usepackage{fancyhdr}
\usepackage{listings} 
\usepackage{xcolor}
\usepackage{geometry}
\usepackage[T1]{fontenc}
\geometry{hmargin=2.5cm,tmargin=1.5cm, bmargin=3cm}
\setlength\hoffset{1em}
\setlength\headheight{32pt}
\setlength\parindent{1em}
\usepackage{hyperref}
\hypersetup{
	colorlinks=true,
	linkcolor=black, 
	urlcolor=[rgb]{0.05,0,0.25}
	}

\lstset{
	numbers=left,
	numberstyle=\tiny\color{gray},
	numbersep=5pt,
	language=bash,
	basicstyle=\ttfamily\footnotesize,
	showstringspaces=false,
	keywordstyle=\bfseries\color{olive},
	% commentstyle=\itshape\color{gray},
	stringstyle=\color{blue},
	breaklines=true,
	breakatwhitespace=true,
	tabsize=4
	}

\titleformat{\section}{\bfseries\Large}{\thesection.}{0.5em}{}
\titleformat{\subsection}{\bfseries\large}{\hspace{.5em}\thesubsection.}{0.5em}{}


%%%%%%%%%%%%%%
%% pied de page et entête
%
\pagestyle{fancy}
\renewcommand{\headrulewidth}{0pt}
\fancyhead{\empty}
\renewcommand{\footrulewidth}{.5pt}
\fancyfoot[L]{\sffamily INF203, Projet Cowsay\empty}
\fancyfoot[C]{\sffamily Shmigelskii A., MELLA G., BASS D.}
\fancyfoot[R]{\sffamily Compte Rendu, Page \thepage}
%%%%%%%%%%%%%%


\begin{document}
\thispagestyle{empty}
\begin{center}

\vspace{10ex}
\huge\sffamily\textbf{Compte Rendu du Projet Cowsay}

\vspace{3ex}
\large


\vspace{3ex}
Aleksandr Shmigelskii, Gabriel Mella, Daniel Bass \\ IMA-05
\end{center}
\vspace{2ex}
{\sffamily \large}

\vspace{.5ex}
\noindent{Enseignant de TD/TP:} Jean-Loup Haberbusch

\vspace{.5ex}
\tableofcontents  
\vspace{2ex}

\newpage
\section{Introduction}

\newpage
\section{Préliminaires}

\begin{table}[ht]
	\centering
	\begin{tabular}{|l|p{5cm}|p{8cm}|}
	\hline
	\textbf{Option} & \textbf{Signification / Effet} & \textbf{Exemple d’usage} \\
	\hline
	\texttt{-b} & Borg mode : la vache aura un aspect “cyborg”. & \verb|cowsay -b "Je suis un Borg"| \\ 
	\hline
	\texttt{-d} & Dead mode : la vache a des yeux « XX ». & \verb|cowsay -d "Aïe. Je ne me sens pas bien"| \\
	\hline
	\texttt{-g} & Greedy mode : la vache a des yeux « \$\$ ». & \verb|cowsay -g "J'adore l'argent"| \\
	\hline
	\texttt{-p} & Paranoïd mode : la vache a des yeux « @@ ». & \verb|cowsay -p "Je suis surveillé..."| \\
	\hline
	\texttt{-s} & Stoned mode : la vache a des yeux « ** ». & \verb|cowsay -s "Coucou..."| \\
	\hline
	\texttt{-t} & Tired mode : la vache a des yeux « -- ». & \verb|cowsay -t "Je suis épuisée..."| \\
	\hline
	\texttt{-w} & Wired mode : la vache a des yeux « OO ». & \verb|cowsay -w "Je ne tiens plus en place"| \\
	\hline
	\texttt{-y} & Youthful mode : la vache a des yeux « .. ». & \verb|cowsay -y "Je suis toute jeune"| \\
	\hline
	\texttt{-e} \emph{eyes} & Personnalise les yeux (2 caractères). & \verb|cowsay -e ^o "Regarde mes yeux"| \\
	\hline
	\texttt{-T} \emph{tongue} & Personnalise la langue (1 ou 2 caractères). & \verb|cowsay -T "U" "Ma langue est sortie"| \\
	\hline
	\texttt{-f} \emph{cowfile} & Utilise un autre dessin ASCII (fichier .cow). & \verb|cowsay -f small "Vraiment petite!"| \\
	\hline
	\texttt{-r} & Choisit une vache au hasard (fichier .cow). & \verb|cowsay -r "Je suis une vache aléatoire."| \\
	\hline
	\texttt{-l} & Liste les vaches définies dans le chemin \textbf{COWPATH} & \verb|cowsay -l| \\
	\hline
	\end{tabular}
	\caption{Principales options de \texttt{cowsay}}
	\label{tab:cowsay-options}
	\end{table}


\newpage
\section{Bash}

\end{document}
